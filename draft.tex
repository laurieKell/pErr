\documentclass[12pt,doublespacing,a4paper]{ouparticle}

%\usepackage[scaled]{helvet}
%\renewcommand\familydefault{\sfdefault} 
%\usepackage[T1]{fontenc}

\usepackage[authoryear]{natbib}
\usepackage{lineno}
\usepackage{pdfpages}
\usepackage{lscape}

%\usepackage{enumerate}
\usepackage{enumitem} 

\usepackage{color}
\definecolor{darkgreen}{rgb}{0.0, 0.5, 0.0}
\definecolor{darkred}{rgb}{0.7, 0.11, 0.11}
\definecolor{darkblue}{rgb}{0,0,0.5}
\definecolor{shadecolor}{rgb}{1,1,0.95}
\definecolor{shade}{rgb}{1,1,0.95}
\definecolor{coilin}{rgb}{1,0,1}

\newcommand{\laurie}{\textcolor{red}}
\newcommand{\coilin}{\textcolor{coilin}}
\newcommand{\hans}{\textcolor{darkblue}}
\newcommand{\alex}{\textcolor{darkred}}
\newcommand{\todo}[1]{\laurie{\textbf{\textit{#1}}}}

\begin{document}

\title{Non-Stationarity in Productivity.}

\author{%
\name{Laurence T. Kell}
\address{Centre for Environmental Policy, Imperial College London, London SW7 1NE}
\email{e-mail Lauriel@seapluslus.co.uk}
\and
\name{}
\address{}
}

\abstract{
The Ecosystem Report Card of the Sub-Committee on Ecosystems includes indicators for assessed species  based on productivity, i.e. trends of biomass or spawning stock biomass relative and fishing mortality or harvest rate relative to Maximum Sustainable Yield reference points. The objective is to assess whether the main target stocks are in a healthy, cautious or critical state and how this has changed over time. Productivity, however, depends on physical and biological processes knowledge of which is important for ecosystem based fisheries management. We therefore evaluate changes in productivity for bigeye and yellowfin tuna in the Atlantic, Indian and Eastern Pacific oceans. 
}

\date{\today}

\keywords{EBFM, productivity, bigeye, yellowfin}
 
\maketitle

\newpage


\linenumbers
\linespread{2}

\section{Introduction}

ICCAT has recently amended its Convention in order to move towards Ecosystem Based Fisheries Management (EBFM) so that the Commission and its Members shall act to: (a) apply the precautionary approach and an ecosystem approach to fisheries management in accordance with relevant internationally agreed standards and, as appropriate, recommended practices and procedures; (b) use the best scientific evidence available; and (c) protect biodiversity in the marine environment.

Currently the main tool developed by the SCRS to implement EBFM is the Ecosystem Report Card (ERC) of the Sub-Committee on Ecosystems (SC-ECO). This includes indicators for assessed species and as the Commission moves towards EBFM there is a need to develop models for stock assessment and management that integrate environmental conditions \citep{travis2014integrating}.  Particularly since growth and mortality, as well as recruitment, may be driven by environmental pressure.

Currently indicators used for assessed species are based on the ratios of biomass or spawning stock biomass (SSB) relative to $B_{MSY}$, and fishing mortality or harvest rate relative to $F_{MSY}$. These indicators are obtained either from age based fisheries stock assessments that integrate changing vulnerability schedules, mean recruitment relationships, and recruitment anomalies, to estimate stock trends and reference points, or biomass dynamic assessments that fit a production function based on population growth rate ($r$), carrying capacity ($K$). In the age based models there is an implicit production function and process error is modelled either as variability in recruitment or selection pattern. In the biomass dynamic models with an explicit production function process error can be modelled explicitly.  

In age based models density dependence is mainly accounted for by the stock recruitment relationship. 
\cite{cury2014resolving}, however, showed that in most cases the stock-recruitment relationship used to estimate productivity and determine reference points, has poor predictive power and the environmental has a larger effect on productivity, a results confirmed by other studies \citep[e.g.][]{szuwalski2014examining} and observed 100 years ago by \cite{hjort1914fluctuations}.

\cite{hilborn2001calculation} recommended looking at patterns of change in surplus production (SP) since these may contain evidence of changes in the growth and mortality components of production, which are typically not represented in models currently used for stock assessment and management. For example in ICCAT assessments growth and natural mortality are commonly assumed not to have varied despite the large changes in the environment and stock biomass seen.

%It is difficult, however, to examine empirical patterns in surplus production for the assessed ICCAT species since there is only limited fisheries independent data. An alternative is to estimate SP using assessment outputs. 

We therefore use the integrated stock assessments conducted using Stock Synthesis 3 \cite[SS3]{methot2005technical} for bigeye and yellowfin tuna stocks in the Atlantic, Indian and Eastern Pacific Oceans to estimate surplus production and explore whether the dynamics are determined by a production function or dynamics are recruitment-driven and influenced by the environment.

\section{Material and Methods}

\cite{walters2008surplus} argued that plots of surplus production ($SP$) vs. biomass ($B$) should be one of the basic pieces of information presented in all stock assessments since the plots provide a check on whether there has been non-stationarity in production processes, i.e. whether similar $B$ levels have exhibited similar SP at different historical times, and importantly for management advice whether predictions of changes in biomass ($B_{t+1} - B_t$) can be made reliably just based on catch and $B_t$. Plots of $SP$ v $B$ therefore provide a summary of stock performance and include effects not necessarily included in stock assessment models. 
 

\begin{itemize}[topsep=0pt,itemsep=0ex,partopsep=0ex,parsep=0ex]
    %\item a direct check on whether predictions of changes in biomass ($B_{t+1} - B_t$) can be made reliably just based on catch and $B_t$, in particular whether similar $B$ levels have exhibited similar SP at different historical times, i.e., whether there has been non-stationarity in production processes. 
    %\item an overall summary of stock performance and include effects not necessarily included in stock assessment models. 
    %\item  a diagnostic for whether stocks have exhibited strong compensatory response, i.e., increase in $S/B$ at low stock sizes. Stocks that have not exhibited such responses have very likely been overfished over the whole period of historical record. In such cases reference points, e.g., $B_0$, should be viewed as highly speculative extrapolations that cannot be justified by available data, no matter how complex the stock assessment models used to analyze those data. 
    %\item  The plots are an important diagnostic for whether stock collapse has been due to overfishing or to antecedent decline in $S$ (that then caused $B$ to decline) attributable to factors other than stock size. 
    %\item ) While fitting SP curves to SP vs. B data is likely to result in highly biased estimates of unfished $B_{0}$, such fitting exercises are not expected to produce badly biased estimates of intrinsic population growth rate ro (Fmax) and hence of “good” target fishing mortality rates. As an example, F = $R_0/2$ is likely a good target exploitation rate that can be put in place without any reference to unfished stock size.
\end{itemize}

We therefore examine the relationships between surplus production and biomass  for bigeye and yellowfin tuna stocks in the Atlantic, Indian and Eastern Pacific Oceans using the results from integrated stock assessments. To do this we estimate annual surplus production as the change in stock size plus catch (i.e. $B_t – B_{t+1} + C_t$). We then plot the resulting time series of $S$ and $B$ to identify patterns of variation in $S$ .  

\section{Results}

\textbf{Figure} \ref{fig:ts} Time series of SSB, biomass and catch relative to  reference points

\textbf{Figure} \ref{fig:sp} Surplus production plotted against biomass, dark to light colour of trajaectory indicates early to late period; clockwise loops in SP caused by positive production anomalies.

\textbf{Figure} \ref{fig:pf} Production functions, with trajectories of catch and biomass; red indicates region of overfishing and overfished.

\textbf{Figure} \ref{fig:pe} Time series of process error.

\textbf{Figure} \ref{fig:pe2} Process error, scaled relative to mean biomass, with regimes


\section{Discussion}

\begin{itemize}[labelindent=\parindent,noitemsep,topsep=0pt,parsep=0pt,partopsep=0pt]
\item The ICCAT and IOTC stocks have shown long term declines, while the IATTC stocks have shown more variability over time.  
 \item 
Independent and positive lognormal recruitment anomalies result in a clockwise loop of successive values when $S$ is plotted against stock size
\item Clockwise cycling was seen for three stocks IATTC bigeye and IOTC and ICCAT yellowfin.
 \item  For IOTC bigeye and yellowfin, larger values of $S$ were seen at low stock biomass. The observed dynamics may also be an indication of model misspecification.
\item \cite{walters} showed that the usual assumptions in age-structured models (stable body growth, no change in natural mortality rate, recruitment following a Beverton–Holt curve) generally implies lower $S$ during population recoveries than during declines because of reduction in mean fecundity.
\end{itemize}

\section{Consequences for EBFM indicators}


\begin{itemize}[labelindent=\parindent,noitemsep,topsep=0pt,parsep=0pt,partopsep=0pt]
\item The presence of clockwise cycling due to recruitment anomalies, implies that future catches are driven by incoming year classes, rather than the production function.
\item There appears to be a suggestion that the stock may be more productive at low stock size that predicted by the implicit production function and reference points derived from them.
\item Need to look at estimates of growth and mortality, e.g. from tagging studies.
\item Possibility of model misspecification.
\item Operating Models used in MSE should be condition on hypotheses related to ecological processes, particularly as this study showed that either the models used for stock assessment are misspecified or do not account for important ecological processes.
\end{itemize}


%In contrast Ecosystem models without age-structured effects typically predict the opposite pattern: higher SP at each stock size during recoveries than declines. Lower production during recoveries is often predicted by ecosystem models with size- and age-structured dynamics, especially when there are cultivation–depensation effects at play (Walters and Kitchell 2001). Higher production rates during recoveries are sometimes predicted by ecosystem models with size- and age-structure effects, but only when such models include ecosystem-scale changes in productivity (regime shifts), such that trophically driven increases in production can drive population increases. 


%Further, it has not been made clear what relationships we should expect between overall biomass and SP during population declines and recoveries and in particular whether age-structured and ecosystem models predict “nonstationarity” in the biomass–production relationship such that production rates exhibited at given biomass levels during recoveries might differ greatly from rates exhibited during declines. A particular concern is whether we should expect slower recoveries (lower SP at each stock size) than we would predict from production rates observed during declines, i.e., should populations recover more slowly than expected from simple dome-shaped production model theory (Hutchings and Reynolds 2004). In the PFMC, they use an age-structured model to predict recoveries so it would automatically incorporate at least recruitment effects caused by reduced mean fecundity in growing populations. Reduced mean fecundity in growing populations occurs because the population age structure is shifted toward younger fish, i.e., there is a proportional scarcity of older, more fecund individuals.

%Here we examine observed relationships between SP rates and stock size for a variety of stocks. We estimate annual SP as change in stock size plus catch (SP = stock next year – stock this year + catch) and plot resulting empirical time series of SP and biomass estimates to look directly at patterns of variation in SP with biomass. We compare these patterns to those predicted by age-structured population models and by ecosystem models with and without age-structure effects. We show that the usual assumptions in age-structured models (stable body growth, no change in natural mortality rate, recruitment following Beverton–Holt curve on average) generally imply lower SP during population recoveries than during declines because of reduction in mean fecundity. However, independent lognormal recruitment anomalies cause the opposite effect, with each positive anomaly resulting in a clockwise loop of successive values when SP is plotted against stock size; further, such anomalies can cause severe bias in estimates of unfished stock size, when a simple SP curve is fitted to the production vs. biomass data. We show that ecosystem models without age-structured effects typically predict the opposite pattern: higher SP at each stock size during recoveries than declines. Lower production during recoveries is often predicted by ecosystem models with size- and age-structured dynamics, especially when there are cultivation–depensation effects at play (Walters and Kitchell 2001). Higher production rates during recoveries are sometimes predicted by ecosystem models with size- and age-structure effects, but only when such models include ecosystem-scale changes in productivity (regime shifts), such that trophically driven increases in production can drive population increases. We recommend that simple plots of SP rate vs. stock size be provided with stock assessment and ecosystem model results in general.

%\citep{{winkler2019jabbasel} Age-structured production models (ASPMs) are often preferred over biomass aggregated surplus production models (SPMs) as the former can track the propagation of cohorts and explicitly account for the effects of selective fishing, even in the absence of reliable size- or age data. Here, we introduce ‘JABBA-Select’, an extension of the JABBA software (Just Another Bayesian Biomass Assessment; Winker et al., 2018), that is able to overcome some of the shortcomings of conventional SPMs and allows a direct comparison to ASPMs. JABBASelect incorporates life history parameters and fishing selectivity and distinguishes between exploitable biomass (used to fit indices given fishery selectivity) and spawning biomass (used to predict surplus production). Applying JABBA-Select involves using an age-structured equilibrium model to convert the input parameters into multivariate normal priors for surplus-production productivity parameters. We illustrate the main elements of JABBA-Select using the stock parameters of South African silver kob (Argyrosomus inodorus, Scienidae) as a case study. This species is exploited by multiple fisheries and was selected as an example of a data moderate fishery that features strong contrast in selectivity over time and across fleets. For proof-of-concept, we use an agestructured simulation framework to compare the performance of JABBA-Select to: 1) a conventional Bayesian state-space model using a Pella-Tomlinson (PT) production function, 2) an ASPM with deterministic recruitment; and 3) an ASPM with stochastic recruitment. The PT model produced biased estimates of relative and absolute spawning biomass trajectories and associated reference points, which by contrast could be fairly accurately estimated by JABBA-Select. JABBA-Select performed at least as well as the ASPMs in accuracy for most of the performance metrics and best characterized the stock status uncertainty. The results indicate that JABBA-Select is able to accurately account for moderate changes in selectivity and fleet dynamics over time and to provide a robust tool for data-moderate stock assessments.

\section{Acknowledgement}


\clearpage
%\bibliography{/home/laurence-kell/Desktop/refs.bib}
%\bibliographystyle{apalike}
\documentclass[12pt,doublespacing,a4paper]{ouparticle}

%\usepackage[scaled]{helvet}
%\renewcommand\familydefault{\sfdefault} 
%\usepackage[T1]{fontenc}

\usepackage[authoryear]{natbib}
\usepackage{lineno}
\usepackage{pdfpages}
\usepackage{lscape}

%\usepackage{enumerate}
\usepackage{enumitem} 

\usepackage{color}
\definecolor{darkgreen}{rgb}{0.0, 0.5, 0.0}
\definecolor{darkred}{rgb}{0.7, 0.11, 0.11}
\definecolor{darkblue}{rgb}{0,0,0.5}
\definecolor{shadecolor}{rgb}{1,1,0.95}
\definecolor{shade}{rgb}{1,1,0.95}
\definecolor{coilin}{rgb}{1,0,1}

\newcommand{\laurie}{\textcolor{red}}
\newcommand{\coilin}{\textcolor{coilin}}
\newcommand{\hans}{\textcolor{darkblue}}
\newcommand{\alex}{\textcolor{darkred}}
\newcommand{\todo}[1]{\laurie{\textbf{\textit{#1}}}}

\begin{document}

\title{Non-Stationarity in Productivity.}

\author{%
\name{Laurence T. Kell}
\address{Centre for Environmental Policy, Imperial College London, London SW7 1NE}
\email{e-mail Lauriel@seapluslus.co.uk}
\and
\name{}
\address{}
}

\abstract{
The Ecosystem Report Card of the Sub-Committee on Ecosystems includes indicators for assessed species  based on productivity, i.e. trends of biomass or spawning stock biomass relative and fishing mortality or harvest rate relative to Maximum Sustainable Yield reference points. The objective is to assess whether the main target stocks are in a healthy, cautious or critical state and how this has changed over time. Productivity, however, depends on physical and biological processes knowledge of which is important for ecosystem based fisheries management. We therefore evaluate changes in productivity for bigeye and yellowfin tuna in the Atlantic, Indian and Eastern Pacific oceans. 
}

\date{\today}

\keywords{EBFM, productivity, bigeye, yellowfin}
 
\maketitle

\newpage


\linenumbers
\linespread{2}

\section{Introduction}

ICCAT has recently amended its Convention in order to move towards Ecosystem Based Fisheries Management (EBFM) so that the Commission and its Members shall act to: (a) apply the precautionary approach and an ecosystem approach to fisheries management in accordance with relevant internationally agreed standards and, as appropriate, recommended practices and procedures; (b) use the best scientific evidence available; and (c) protect biodiversity in the marine environment.

Currently the main tool developed by the SCRS to implement EBFM is the Ecosystem Report Card (ERC) of the Sub-Committee on Ecosystems (SC-ECO). This includes indicators for assessed species and as the Commission moves towards EBFM there is a need to develop models for stock assessment and management that integrate environmental conditions \citep{travis2014integrating}.  Particularly since growth and mortality, as well as recruitment, may be driven by environmental pressure.

Currently indicators used for assessed species are based on the ratios of biomass or spawning stock biomass (SSB) relative to $B_{MSY}$, and fishing mortality or harvest rate relative to $F_{MSY}$. These indicators are obtained either from age based fisheries stock assessments that integrate changing vulnerability schedules, mean recruitment relationships, and recruitment anomalies, to estimate stock trends and reference points, or biomass dynamic assessments that fit a production function based on population growth rate ($r$), carrying capacity ($K$). In the age based models there is an implicit production function and process error is modelled either as variability in recruitment or selection pattern. In the biomass dynamic models with an explicit production function process error can be modelled explicitly.  

In age based models density dependence is mainly accounted for by the stock recruitment relationship. 
\cite{cury2014resolving}, however, showed that in most cases the stock-recruitment relationship used to estimate productivity and determine reference points, has poor predictive power and the environmental has a larger effect on productivity, a results confirmed by other studies \citep[e.g.][]{szuwalski2014examining} and observed 100 years ago by \cite{hjort1914fluctuations}.

\cite{hilborn2001calculation} recommended looking at patterns of change in surplus production (SP) since these may contain evidence of changes in the growth and mortality components of production, which are typically not represented in models currently used for stock assessment and management. For example in ICCAT assessments growth and natural mortality are commonly assumed not to have varied despite the large changes in the environment and stock biomass seen.

%It is difficult, however, to examine empirical patterns in surplus production for the assessed ICCAT species since there is only limited fisheries independent data. An alternative is to estimate SP using assessment outputs. 

We therefore use the integrated stock assessments conducted using Stock Synthesis 3 \cite[SS3]{methot2005technical} for bigeye and yellowfin tuna stocks in the Atlantic, Indian and Eastern Pacific Oceans to estimate surplus production and explore whether the dynamics are determined by a production function or dynamics are recruitment-driven and influenced by the environment.

\section{Material and Methods}

\cite{walters2008surplus} argued that plots of surplus production ($SP$) vs. biomass ($B$) should be one of the basic pieces of information presented in all stock assessments since the plots provide a check on whether there has been non-stationarity in production processes, i.e. whether similar $B$ levels have exhibited similar SP at different historical times, and importantly for management advice whether predictions of changes in biomass ($B_{t+1} - B_t$) can be made reliably just based on catch and $B_t$. Plots of $SP$ v $B$ therefore provide a summary of stock performance and include effects not necessarily included in stock assessment models. 
 

\begin{itemize}[topsep=0pt,itemsep=0ex,partopsep=0ex,parsep=0ex]
    %\item a direct check on whether predictions of changes in biomass ($B_{t+1} - B_t$) can be made reliably just based on catch and $B_t$, in particular whether similar $B$ levels have exhibited similar SP at different historical times, i.e., whether there has been non-stationarity in production processes. 
    %\item an overall summary of stock performance and include effects not necessarily included in stock assessment models. 
    %\item  a diagnostic for whether stocks have exhibited strong compensatory response, i.e., increase in $S/B$ at low stock sizes. Stocks that have not exhibited such responses have very likely been overfished over the whole period of historical record. In such cases reference points, e.g., $B_0$, should be viewed as highly speculative extrapolations that cannot be justified by available data, no matter how complex the stock assessment models used to analyze those data. 
    %\item  The plots are an important diagnostic for whether stock collapse has been due to overfishing or to antecedent decline in $S$ (that then caused $B$ to decline) attributable to factors other than stock size. 
    %\item ) While fitting SP curves to SP vs. B data is likely to result in highly biased estimates of unfished $B_{0}$, such fitting exercises are not expected to produce badly biased estimates of intrinsic population growth rate ro (Fmax) and hence of “good” target fishing mortality rates. As an example, F = $R_0/2$ is likely a good target exploitation rate that can be put in place without any reference to unfished stock size.
\end{itemize}

We therefore examine the relationships between surplus production and biomass  for bigeye and yellowfin tuna stocks in the Atlantic, Indian and Eastern Pacific Oceans using the results from integrated stock assessments. To do this we estimate annual surplus production as the change in stock size plus catch (i.e. $B_t – B_{t+1} + C_t$). We then plot the resulting time series of $S$ and $B$ to identify patterns of variation in $S$ .  

\section{Results}

\textbf{Figure} \ref{fig:ts} Time series of SSB, biomass and catch relative to  reference points

\textbf{Figure} \ref{fig:sp} Surplus production plotted against biomass, dark to light colour of trajaectory indicates early to late period; clockwise loops in SP caused by positive production anomalies.

\textbf{Figure} \ref{fig:pf} Production functions, with trajectories of catch and biomass; red indicates region of overfishing and overfished.

\textbf{Figure} \ref{fig:pe} Time series of process error.

\textbf{Figure} \ref{fig:pe2} Process error, scaled relative to mean biomass, with regimes


\section{Discussion}

\begin{itemize}[labelindent=\parindent,noitemsep,topsep=0pt,parsep=0pt,partopsep=0pt]
\item The ICCAT and IOTC stocks have shown long term declines, while the IATTC stocks have shown more variability over time.  
 \item 
Independent and positive lognormal recruitment anomalies result in a clockwise loop of successive values when $S$ is plotted against stock size
\item Clockwise cycling was seen for three stocks IATTC bigeye and IOTC and ICCAT yellowfin.
 \item  For IOTC bigeye and yellowfin, larger values of $S$ were seen at low stock biomass. The observed dynamics may also be an indication of model misspecification.
\item \cite{walters} showed that the usual assumptions in age-structured models (stable body growth, no change in natural mortality rate, recruitment following a Beverton–Holt curve) generally implies lower $S$ during population recoveries than during declines because of reduction in mean fecundity.
\end{itemize}

\section{Consequences for EBFM indicators}


\begin{itemize}[labelindent=\parindent,noitemsep,topsep=0pt,parsep=0pt,partopsep=0pt]
\item The presence of clockwise cycling due to recruitment anomalies, implies that future catches are driven by incoming year classes, rather than the production function.
\item There appears to be a suggestion that the stock may be more productive at low stock size that predicted by the implicit production function and reference points derived from them.
\item Need to look at estimates of growth and mortality, e.g. from tagging studies.
\item Possibility of model misspecification.
\item Operating Models used in MSE should be condition on hypotheses related to ecological processes, particularly as this study showed that either the models used for stock assessment are misspecified or do not account for important ecological processes.
\end{itemize}


%In contrast Ecosystem models without age-structured effects typically predict the opposite pattern: higher SP at each stock size during recoveries than declines. Lower production during recoveries is often predicted by ecosystem models with size- and age-structured dynamics, especially when there are cultivation–depensation effects at play (Walters and Kitchell 2001). Higher production rates during recoveries are sometimes predicted by ecosystem models with size- and age-structure effects, but only when such models include ecosystem-scale changes in productivity (regime shifts), such that trophically driven increases in production can drive population increases. 


%Further, it has not been made clear what relationships we should expect between overall biomass and SP during population declines and recoveries and in particular whether age-structured and ecosystem models predict “nonstationarity” in the biomass–production relationship such that production rates exhibited at given biomass levels during recoveries might differ greatly from rates exhibited during declines. A particular concern is whether we should expect slower recoveries (lower SP at each stock size) than we would predict from production rates observed during declines, i.e., should populations recover more slowly than expected from simple dome-shaped production model theory (Hutchings and Reynolds 2004). In the PFMC, they use an age-structured model to predict recoveries so it would automatically incorporate at least recruitment effects caused by reduced mean fecundity in growing populations. Reduced mean fecundity in growing populations occurs because the population age structure is shifted toward younger fish, i.e., there is a proportional scarcity of older, more fecund individuals.

%Here we examine observed relationships between SP rates and stock size for a variety of stocks. We estimate annual SP as change in stock size plus catch (SP = stock next year – stock this year + catch) and plot resulting empirical time series of SP and biomass estimates to look directly at patterns of variation in SP with biomass. We compare these patterns to those predicted by age-structured population models and by ecosystem models with and without age-structure effects. We show that the usual assumptions in age-structured models (stable body growth, no change in natural mortality rate, recruitment following Beverton–Holt curve on average) generally imply lower SP during population recoveries than during declines because of reduction in mean fecundity. However, independent lognormal recruitment anomalies cause the opposite effect, with each positive anomaly resulting in a clockwise loop of successive values when SP is plotted against stock size; further, such anomalies can cause severe bias in estimates of unfished stock size, when a simple SP curve is fitted to the production vs. biomass data. We show that ecosystem models without age-structured effects typically predict the opposite pattern: higher SP at each stock size during recoveries than declines. Lower production during recoveries is often predicted by ecosystem models with size- and age-structured dynamics, especially when there are cultivation–depensation effects at play (Walters and Kitchell 2001). Higher production rates during recoveries are sometimes predicted by ecosystem models with size- and age-structure effects, but only when such models include ecosystem-scale changes in productivity (regime shifts), such that trophically driven increases in production can drive population increases. We recommend that simple plots of SP rate vs. stock size be provided with stock assessment and ecosystem model results in general.

%\citep{{winkler2019jabbasel} Age-structured production models (ASPMs) are often preferred over biomass aggregated surplus production models (SPMs) as the former can track the propagation of cohorts and explicitly account for the effects of selective fishing, even in the absence of reliable size- or age data. Here, we introduce ‘JABBA-Select’, an extension of the JABBA software (Just Another Bayesian Biomass Assessment; Winker et al., 2018), that is able to overcome some of the shortcomings of conventional SPMs and allows a direct comparison to ASPMs. JABBASelect incorporates life history parameters and fishing selectivity and distinguishes between exploitable biomass (used to fit indices given fishery selectivity) and spawning biomass (used to predict surplus production). Applying JABBA-Select involves using an age-structured equilibrium model to convert the input parameters into multivariate normal priors for surplus-production productivity parameters. We illustrate the main elements of JABBA-Select using the stock parameters of South African silver kob (Argyrosomus inodorus, Scienidae) as a case study. This species is exploited by multiple fisheries and was selected as an example of a data moderate fishery that features strong contrast in selectivity over time and across fleets. For proof-of-concept, we use an agestructured simulation framework to compare the performance of JABBA-Select to: 1) a conventional Bayesian state-space model using a Pella-Tomlinson (PT) production function, 2) an ASPM with deterministic recruitment; and 3) an ASPM with stochastic recruitment. The PT model produced biased estimates of relative and absolute spawning biomass trajectories and associated reference points, which by contrast could be fairly accurately estimated by JABBA-Select. JABBA-Select performed at least as well as the ASPMs in accuracy for most of the performance metrics and best characterized the stock status uncertainty. The results indicate that JABBA-Select is able to accurately account for moderate changes in selectivity and fleet dynamics over time and to provide a robust tool for data-moderate stock assessments.

\section{Acknowledgement}


\clearpage
%\bibliography{/home/laurence-kell/Desktop/refs.bib}
%\bibliographystyle{apalike}
\documentclass[12pt,doublespacing,a4paper]{ouparticle}

%\usepackage[scaled]{helvet}
%\renewcommand\familydefault{\sfdefault} 
%\usepackage[T1]{fontenc}

\usepackage[authoryear]{natbib}
\usepackage{lineno}
\usepackage{pdfpages}
\usepackage{lscape}

%\usepackage{enumerate}
\usepackage{enumitem} 

\usepackage{color}
\definecolor{darkgreen}{rgb}{0.0, 0.5, 0.0}
\definecolor{darkred}{rgb}{0.7, 0.11, 0.11}
\definecolor{darkblue}{rgb}{0,0,0.5}
\definecolor{shadecolor}{rgb}{1,1,0.95}
\definecolor{shade}{rgb}{1,1,0.95}
\definecolor{coilin}{rgb}{1,0,1}

\newcommand{\laurie}{\textcolor{red}}
\newcommand{\coilin}{\textcolor{coilin}}
\newcommand{\hans}{\textcolor{darkblue}}
\newcommand{\alex}{\textcolor{darkred}}
\newcommand{\todo}[1]{\laurie{\textbf{\textit{#1}}}}

\begin{document}

\title{Non-Stationarity in Productivity.}

\author{%
\name{Laurence T. Kell}
\address{Centre for Environmental Policy, Imperial College London, London SW7 1NE}
\email{e-mail Lauriel@seapluslus.co.uk}
\and
\name{}
\address{}
}

\abstract{
The Ecosystem Report Card of the Sub-Committee on Ecosystems includes indicators for assessed species  based on productivity, i.e. trends of biomass or spawning stock biomass relative and fishing mortality or harvest rate relative to Maximum Sustainable Yield reference points. The objective is to assess whether the main target stocks are in a healthy, cautious or critical state and how this has changed over time. Productivity, however, depends on physical and biological processes knowledge of which is important for ecosystem based fisheries management. We therefore evaluate changes in productivity for bigeye and yellowfin tuna in the Atlantic, Indian and Eastern Pacific oceans. 
}

\date{\today}

\keywords{EBFM, productivity, bigeye, yellowfin}
 
\maketitle

\newpage


\linenumbers
\linespread{2}

\section{Introduction}

ICCAT has recently amended its Convention in order to move towards Ecosystem Based Fisheries Management (EBFM) so that the Commission and its Members shall act to: (a) apply the precautionary approach and an ecosystem approach to fisheries management in accordance with relevant internationally agreed standards and, as appropriate, recommended practices and procedures; (b) use the best scientific evidence available; and (c) protect biodiversity in the marine environment.

Currently the main tool developed by the SCRS to implement EBFM is the Ecosystem Report Card (ERC) of the Sub-Committee on Ecosystems (SC-ECO). This includes indicators for assessed species and as the Commission moves towards EBFM there is a need to develop models for stock assessment and management that integrate environmental conditions \citep{travis2014integrating}.  Particularly since growth and mortality, as well as recruitment, may be driven by environmental pressure.

Currently indicators used for assessed species are based on the ratios of biomass or spawning stock biomass (SSB) relative to $B_{MSY}$, and fishing mortality or harvest rate relative to $F_{MSY}$. These indicators are obtained either from age based fisheries stock assessments that integrate changing vulnerability schedules, mean recruitment relationships, and recruitment anomalies, to estimate stock trends and reference points, or biomass dynamic assessments that fit a production function based on population growth rate ($r$), carrying capacity ($K$). In the age based models there is an implicit production function and process error is modelled either as variability in recruitment or selection pattern. In the biomass dynamic models with an explicit production function process error can be modelled explicitly.  

In age based models density dependence is mainly accounted for by the stock recruitment relationship. 
\cite{cury2014resolving}, however, showed that in most cases the stock-recruitment relationship used to estimate productivity and determine reference points, has poor predictive power and the environmental has a larger effect on productivity, a results confirmed by other studies \citep[e.g.][]{szuwalski2014examining} and observed 100 years ago by \cite{hjort1914fluctuations}.

\cite{hilborn2001calculation} recommended looking at patterns of change in surplus production (SP) since these may contain evidence of changes in the growth and mortality components of production, which are typically not represented in models currently used for stock assessment and management. For example in ICCAT assessments growth and natural mortality are commonly assumed not to have varied despite the large changes in the environment and stock biomass seen.

%It is difficult, however, to examine empirical patterns in surplus production for the assessed ICCAT species since there is only limited fisheries independent data. An alternative is to estimate SP using assessment outputs. 

We therefore use the integrated stock assessments conducted using Stock Synthesis 3 \cite[SS3]{methot2005technical} for bigeye and yellowfin tuna stocks in the Atlantic, Indian and Eastern Pacific Oceans to estimate surplus production and explore whether the dynamics are determined by a production function or dynamics are recruitment-driven and influenced by the environment.

\section{Material and Methods}

\cite{walters2008surplus} argued that plots of surplus production ($SP$) vs. biomass ($B$) should be one of the basic pieces of information presented in all stock assessments since the plots provide a check on whether there has been non-stationarity in production processes, i.e. whether similar $B$ levels have exhibited similar SP at different historical times, and importantly for management advice whether predictions of changes in biomass ($B_{t+1} - B_t$) can be made reliably just based on catch and $B_t$. Plots of $SP$ v $B$ therefore provide a summary of stock performance and include effects not necessarily included in stock assessment models. 
 

\begin{itemize}[topsep=0pt,itemsep=0ex,partopsep=0ex,parsep=0ex]
    %\item a direct check on whether predictions of changes in biomass ($B_{t+1} - B_t$) can be made reliably just based on catch and $B_t$, in particular whether similar $B$ levels have exhibited similar SP at different historical times, i.e., whether there has been non-stationarity in production processes. 
    %\item an overall summary of stock performance and include effects not necessarily included in stock assessment models. 
    %\item  a diagnostic for whether stocks have exhibited strong compensatory response, i.e., increase in $S/B$ at low stock sizes. Stocks that have not exhibited such responses have very likely been overfished over the whole period of historical record. In such cases reference points, e.g., $B_0$, should be viewed as highly speculative extrapolations that cannot be justified by available data, no matter how complex the stock assessment models used to analyze those data. 
    %\item  The plots are an important diagnostic for whether stock collapse has been due to overfishing or to antecedent decline in $S$ (that then caused $B$ to decline) attributable to factors other than stock size. 
    %\item ) While fitting SP curves to SP vs. B data is likely to result in highly biased estimates of unfished $B_{0}$, such fitting exercises are not expected to produce badly biased estimates of intrinsic population growth rate ro (Fmax) and hence of “good” target fishing mortality rates. As an example, F = $R_0/2$ is likely a good target exploitation rate that can be put in place without any reference to unfished stock size.
\end{itemize}

We therefore examine the relationships between surplus production and biomass  for bigeye and yellowfin tuna stocks in the Atlantic, Indian and Eastern Pacific Oceans using the results from integrated stock assessments. To do this we estimate annual surplus production as the change in stock size plus catch (i.e. $B_t – B_{t+1} + C_t$). We then plot the resulting time series of $S$ and $B$ to identify patterns of variation in $S$ .  

\section{Results}

\textbf{Figure} \ref{fig:ts} Time series of SSB, biomass and catch relative to  reference points

\textbf{Figure} \ref{fig:sp} Surplus production plotted against biomass, dark to light colour of trajaectory indicates early to late period; clockwise loops in SP caused by positive production anomalies.

\textbf{Figure} \ref{fig:pf} Production functions, with trajectories of catch and biomass; red indicates region of overfishing and overfished.

\textbf{Figure} \ref{fig:pe} Time series of process error.

\textbf{Figure} \ref{fig:pe2} Process error, scaled relative to mean biomass, with regimes


\section{Discussion}

\begin{itemize}[labelindent=\parindent,noitemsep,topsep=0pt,parsep=0pt,partopsep=0pt]
\item The ICCAT and IOTC stocks have shown long term declines, while the IATTC stocks have shown more variability over time.  
 \item 
Independent and positive lognormal recruitment anomalies result in a clockwise loop of successive values when $S$ is plotted against stock size
\item Clockwise cycling was seen for three stocks IATTC bigeye and IOTC and ICCAT yellowfin.
 \item  For IOTC bigeye and yellowfin, larger values of $S$ were seen at low stock biomass. The observed dynamics may also be an indication of model misspecification.
\item \cite{walters} showed that the usual assumptions in age-structured models (stable body growth, no change in natural mortality rate, recruitment following a Beverton–Holt curve) generally implies lower $S$ during population recoveries than during declines because of reduction in mean fecundity.
\end{itemize}

\section{Consequences for EBFM indicators}


\begin{itemize}[labelindent=\parindent,noitemsep,topsep=0pt,parsep=0pt,partopsep=0pt]
\item The presence of clockwise cycling due to recruitment anomalies, implies that future catches are driven by incoming year classes, rather than the production function.
\item There appears to be a suggestion that the stock may be more productive at low stock size that predicted by the implicit production function and reference points derived from them.
\item Need to look at estimates of growth and mortality, e.g. from tagging studies.
\item Possibility of model misspecification.
\item Operating Models used in MSE should be condition on hypotheses related to ecological processes, particularly as this study showed that either the models used for stock assessment are misspecified or do not account for important ecological processes.
\end{itemize}


%In contrast Ecosystem models without age-structured effects typically predict the opposite pattern: higher SP at each stock size during recoveries than declines. Lower production during recoveries is often predicted by ecosystem models with size- and age-structured dynamics, especially when there are cultivation–depensation effects at play (Walters and Kitchell 2001). Higher production rates during recoveries are sometimes predicted by ecosystem models with size- and age-structure effects, but only when such models include ecosystem-scale changes in productivity (regime shifts), such that trophically driven increases in production can drive population increases. 


%Further, it has not been made clear what relationships we should expect between overall biomass and SP during population declines and recoveries and in particular whether age-structured and ecosystem models predict “nonstationarity” in the biomass–production relationship such that production rates exhibited at given biomass levels during recoveries might differ greatly from rates exhibited during declines. A particular concern is whether we should expect slower recoveries (lower SP at each stock size) than we would predict from production rates observed during declines, i.e., should populations recover more slowly than expected from simple dome-shaped production model theory (Hutchings and Reynolds 2004). In the PFMC, they use an age-structured model to predict recoveries so it would automatically incorporate at least recruitment effects caused by reduced mean fecundity in growing populations. Reduced mean fecundity in growing populations occurs because the population age structure is shifted toward younger fish, i.e., there is a proportional scarcity of older, more fecund individuals.

%Here we examine observed relationships between SP rates and stock size for a variety of stocks. We estimate annual SP as change in stock size plus catch (SP = stock next year – stock this year + catch) and plot resulting empirical time series of SP and biomass estimates to look directly at patterns of variation in SP with biomass. We compare these patterns to those predicted by age-structured population models and by ecosystem models with and without age-structure effects. We show that the usual assumptions in age-structured models (stable body growth, no change in natural mortality rate, recruitment following Beverton–Holt curve on average) generally imply lower SP during population recoveries than during declines because of reduction in mean fecundity. However, independent lognormal recruitment anomalies cause the opposite effect, with each positive anomaly resulting in a clockwise loop of successive values when SP is plotted against stock size; further, such anomalies can cause severe bias in estimates of unfished stock size, when a simple SP curve is fitted to the production vs. biomass data. We show that ecosystem models without age-structured effects typically predict the opposite pattern: higher SP at each stock size during recoveries than declines. Lower production during recoveries is often predicted by ecosystem models with size- and age-structured dynamics, especially when there are cultivation–depensation effects at play (Walters and Kitchell 2001). Higher production rates during recoveries are sometimes predicted by ecosystem models with size- and age-structure effects, but only when such models include ecosystem-scale changes in productivity (regime shifts), such that trophically driven increases in production can drive population increases. We recommend that simple plots of SP rate vs. stock size be provided with stock assessment and ecosystem model results in general.

%\citep{{winkler2019jabbasel} Age-structured production models (ASPMs) are often preferred over biomass aggregated surplus production models (SPMs) as the former can track the propagation of cohorts and explicitly account for the effects of selective fishing, even in the absence of reliable size- or age data. Here, we introduce ‘JABBA-Select’, an extension of the JABBA software (Just Another Bayesian Biomass Assessment; Winker et al., 2018), that is able to overcome some of the shortcomings of conventional SPMs and allows a direct comparison to ASPMs. JABBASelect incorporates life history parameters and fishing selectivity and distinguishes between exploitable biomass (used to fit indices given fishery selectivity) and spawning biomass (used to predict surplus production). Applying JABBA-Select involves using an age-structured equilibrium model to convert the input parameters into multivariate normal priors for surplus-production productivity parameters. We illustrate the main elements of JABBA-Select using the stock parameters of South African silver kob (Argyrosomus inodorus, Scienidae) as a case study. This species is exploited by multiple fisheries and was selected as an example of a data moderate fishery that features strong contrast in selectivity over time and across fleets. For proof-of-concept, we use an agestructured simulation framework to compare the performance of JABBA-Select to: 1) a conventional Bayesian state-space model using a Pella-Tomlinson (PT) production function, 2) an ASPM with deterministic recruitment; and 3) an ASPM with stochastic recruitment. The PT model produced biased estimates of relative and absolute spawning biomass trajectories and associated reference points, which by contrast could be fairly accurately estimated by JABBA-Select. JABBA-Select performed at least as well as the ASPMs in accuracy for most of the performance metrics and best characterized the stock status uncertainty. The results indicate that JABBA-Select is able to accurately account for moderate changes in selectivity and fleet dynamics over time and to provide a robust tool for data-moderate stock assessments.

\section{Acknowledgement}


\clearpage
%\bibliography{/home/laurence-kell/Desktop/refs.bib}
%\bibliographystyle{apalike}
\documentclass[12pt,doublespacing,a4paper]{ouparticle}

%\usepackage[scaled]{helvet}
%\renewcommand\familydefault{\sfdefault} 
%\usepackage[T1]{fontenc}

\usepackage[authoryear]{natbib}
\usepackage{lineno}
\usepackage{pdfpages}
\usepackage{lscape}

%\usepackage{enumerate}
\usepackage{enumitem} 

\usepackage{color}
\definecolor{darkgreen}{rgb}{0.0, 0.5, 0.0}
\definecolor{darkred}{rgb}{0.7, 0.11, 0.11}
\definecolor{darkblue}{rgb}{0,0,0.5}
\definecolor{shadecolor}{rgb}{1,1,0.95}
\definecolor{shade}{rgb}{1,1,0.95}
\definecolor{coilin}{rgb}{1,0,1}

\newcommand{\laurie}{\textcolor{red}}
\newcommand{\coilin}{\textcolor{coilin}}
\newcommand{\hans}{\textcolor{darkblue}}
\newcommand{\alex}{\textcolor{darkred}}
\newcommand{\todo}[1]{\laurie{\textbf{\textit{#1}}}}

\begin{document}

\title{Non-Stationarity in Productivity.}

\author{%
\name{Laurence T. Kell}
\address{Centre for Environmental Policy, Imperial College London, London SW7 1NE}
\email{e-mail Lauriel@seapluslus.co.uk}
\and
\name{}
\address{}
}

\abstract{
The Ecosystem Report Card of the Sub-Committee on Ecosystems includes indicators for assessed species  based on productivity, i.e. trends of biomass or spawning stock biomass relative and fishing mortality or harvest rate relative to Maximum Sustainable Yield reference points. The objective is to assess whether the main target stocks are in a healthy, cautious or critical state and how this has changed over time. Productivity, however, depends on physical and biological processes knowledge of which is important for ecosystem based fisheries management. We therefore evaluate changes in productivity for bigeye and yellowfin tuna in the Atlantic, Indian and Eastern Pacific oceans. 
}

\date{\today}

\keywords{EBFM, productivity, bigeye, yellowfin}
 
\maketitle

\newpage


\linenumbers
\linespread{2}

\section{Introduction}

ICCAT has recently amended its Convention in order to move towards Ecosystem Based Fisheries Management (EBFM) so that the Commission and its Members shall act to: (a) apply the precautionary approach and an ecosystem approach to fisheries management in accordance with relevant internationally agreed standards and, as appropriate, recommended practices and procedures; (b) use the best scientific evidence available; and (c) protect biodiversity in the marine environment.

Currently the main tool developed by the SCRS to implement EBFM is the Ecosystem Report Card (ERC) of the Sub-Committee on Ecosystems (SC-ECO). This includes indicators for assessed species and as the Commission moves towards EBFM there is a need to develop models for stock assessment and management that integrate environmental conditions \citep{travis2014integrating}.  Particularly since growth and mortality, as well as recruitment, may be driven by environmental pressure.

Currently indicators used for assessed species are based on the ratios of biomass or spawning stock biomass (SSB) relative to $B_{MSY}$, and fishing mortality or harvest rate relative to $F_{MSY}$. These indicators are obtained either from age based fisheries stock assessments that integrate changing vulnerability schedules, mean recruitment relationships, and recruitment anomalies, to estimate stock trends and reference points, or biomass dynamic assessments that fit a production function based on population growth rate ($r$), carrying capacity ($K$). In the age based models there is an implicit production function and process error is modelled either as variability in recruitment or selection pattern. In the biomass dynamic models with an explicit production function process error can be modelled explicitly.  

In age based models density dependence is mainly accounted for by the stock recruitment relationship. 
\cite{cury2014resolving}, however, showed that in most cases the stock-recruitment relationship used to estimate productivity and determine reference points, has poor predictive power and the environmental has a larger effect on productivity, a results confirmed by other studies \citep[e.g.][]{szuwalski2014examining} and observed 100 years ago by \cite{hjort1914fluctuations}.

\cite{hilborn2001calculation} recommended looking at patterns of change in surplus production (SP) since these may contain evidence of changes in the growth and mortality components of production, which are typically not represented in models currently used for stock assessment and management. For example in ICCAT assessments growth and natural mortality are commonly assumed not to have varied despite the large changes in the environment and stock biomass seen.

%It is difficult, however, to examine empirical patterns in surplus production for the assessed ICCAT species since there is only limited fisheries independent data. An alternative is to estimate SP using assessment outputs. 

We therefore use the integrated stock assessments conducted using Stock Synthesis 3 \cite[SS3]{methot2005technical} for bigeye and yellowfin tuna stocks in the Atlantic, Indian and Eastern Pacific Oceans to estimate surplus production and explore whether the dynamics are determined by a production function or dynamics are recruitment-driven and influenced by the environment.

\section{Material and Methods}

\cite{walters2008surplus} argued that plots of surplus production ($SP$) vs. biomass ($B$) should be one of the basic pieces of information presented in all stock assessments since the plots provide a check on whether there has been non-stationarity in production processes, i.e. whether similar $B$ levels have exhibited similar SP at different historical times, and importantly for management advice whether predictions of changes in biomass ($B_{t+1} - B_t$) can be made reliably just based on catch and $B_t$. Plots of $SP$ v $B$ therefore provide a summary of stock performance and include effects not necessarily included in stock assessment models. 
 

\begin{itemize}[topsep=0pt,itemsep=0ex,partopsep=0ex,parsep=0ex]
    %\item a direct check on whether predictions of changes in biomass ($B_{t+1} - B_t$) can be made reliably just based on catch and $B_t$, in particular whether similar $B$ levels have exhibited similar SP at different historical times, i.e., whether there has been non-stationarity in production processes. 
    %\item an overall summary of stock performance and include effects not necessarily included in stock assessment models. 
    %\item  a diagnostic for whether stocks have exhibited strong compensatory response, i.e., increase in $S/B$ at low stock sizes. Stocks that have not exhibited such responses have very likely been overfished over the whole period of historical record. In such cases reference points, e.g., $B_0$, should be viewed as highly speculative extrapolations that cannot be justified by available data, no matter how complex the stock assessment models used to analyze those data. 
    %\item  The plots are an important diagnostic for whether stock collapse has been due to overfishing or to antecedent decline in $S$ (that then caused $B$ to decline) attributable to factors other than stock size. 
    %\item ) While fitting SP curves to SP vs. B data is likely to result in highly biased estimates of unfished $B_{0}$, such fitting exercises are not expected to produce badly biased estimates of intrinsic population growth rate ro (Fmax) and hence of “good” target fishing mortality rates. As an example, F = $R_0/2$ is likely a good target exploitation rate that can be put in place without any reference to unfished stock size.
\end{itemize}

We therefore examine the relationships between surplus production and biomass  for bigeye and yellowfin tuna stocks in the Atlantic, Indian and Eastern Pacific Oceans using the results from integrated stock assessments. To do this we estimate annual surplus production as the change in stock size plus catch (i.e. $B_t – B_{t+1} + C_t$). We then plot the resulting time series of $S$ and $B$ to identify patterns of variation in $S$ .  

\section{Results}

\textbf{Figure} \ref{fig:ts} Time series of SSB, biomass and catch relative to  reference points

\textbf{Figure} \ref{fig:sp} Surplus production plotted against biomass, dark to light colour of trajaectory indicates early to late period; clockwise loops in SP caused by positive production anomalies.

\textbf{Figure} \ref{fig:pf} Production functions, with trajectories of catch and biomass; red indicates region of overfishing and overfished.

\textbf{Figure} \ref{fig:pe} Time series of process error.

\textbf{Figure} \ref{fig:pe2} Process error, scaled relative to mean biomass, with regimes


\section{Discussion}

\begin{itemize}[labelindent=\parindent,noitemsep,topsep=0pt,parsep=0pt,partopsep=0pt]
\item The ICCAT and IOTC stocks have shown long term declines, while the IATTC stocks have shown more variability over time.  
 \item 
Independent and positive lognormal recruitment anomalies result in a clockwise loop of successive values when $S$ is plotted against stock size
\item Clockwise cycling was seen for three stocks IATTC bigeye and IOTC and ICCAT yellowfin.
 \item  For IOTC bigeye and yellowfin, larger values of $S$ were seen at low stock biomass. The observed dynamics may also be an indication of model misspecification.
\item \cite{walters} showed that the usual assumptions in age-structured models (stable body growth, no change in natural mortality rate, recruitment following a Beverton–Holt curve) generally implies lower $S$ during population recoveries than during declines because of reduction in mean fecundity.
\end{itemize}

\section{Consequences for EBFM indicators}


\begin{itemize}[labelindent=\parindent,noitemsep,topsep=0pt,parsep=0pt,partopsep=0pt]
\item The presence of clockwise cycling due to recruitment anomalies, implies that future catches are driven by incoming year classes, rather than the production function.
\item There appears to be a suggestion that the stock may be more productive at low stock size that predicted by the implicit production function and reference points derived from them.
\item Need to look at estimates of growth and mortality, e.g. from tagging studies.
\item Possibility of model misspecification.
\item Operating Models used in MSE should be condition on hypotheses related to ecological processes, particularly as this study showed that either the models used for stock assessment are misspecified or do not account for important ecological processes.
\end{itemize}


%In contrast Ecosystem models without age-structured effects typically predict the opposite pattern: higher SP at each stock size during recoveries than declines. Lower production during recoveries is often predicted by ecosystem models with size- and age-structured dynamics, especially when there are cultivation–depensation effects at play (Walters and Kitchell 2001). Higher production rates during recoveries are sometimes predicted by ecosystem models with size- and age-structure effects, but only when such models include ecosystem-scale changes in productivity (regime shifts), such that trophically driven increases in production can drive population increases. 


%Further, it has not been made clear what relationships we should expect between overall biomass and SP during population declines and recoveries and in particular whether age-structured and ecosystem models predict “nonstationarity” in the biomass–production relationship such that production rates exhibited at given biomass levels during recoveries might differ greatly from rates exhibited during declines. A particular concern is whether we should expect slower recoveries (lower SP at each stock size) than we would predict from production rates observed during declines, i.e., should populations recover more slowly than expected from simple dome-shaped production model theory (Hutchings and Reynolds 2004). In the PFMC, they use an age-structured model to predict recoveries so it would automatically incorporate at least recruitment effects caused by reduced mean fecundity in growing populations. Reduced mean fecundity in growing populations occurs because the population age structure is shifted toward younger fish, i.e., there is a proportional scarcity of older, more fecund individuals.

%Here we examine observed relationships between SP rates and stock size for a variety of stocks. We estimate annual SP as change in stock size plus catch (SP = stock next year – stock this year + catch) and plot resulting empirical time series of SP and biomass estimates to look directly at patterns of variation in SP with biomass. We compare these patterns to those predicted by age-structured population models and by ecosystem models with and without age-structure effects. We show that the usual assumptions in age-structured models (stable body growth, no change in natural mortality rate, recruitment following Beverton–Holt curve on average) generally imply lower SP during population recoveries than during declines because of reduction in mean fecundity. However, independent lognormal recruitment anomalies cause the opposite effect, with each positive anomaly resulting in a clockwise loop of successive values when SP is plotted against stock size; further, such anomalies can cause severe bias in estimates of unfished stock size, when a simple SP curve is fitted to the production vs. biomass data. We show that ecosystem models without age-structured effects typically predict the opposite pattern: higher SP at each stock size during recoveries than declines. Lower production during recoveries is often predicted by ecosystem models with size- and age-structured dynamics, especially when there are cultivation–depensation effects at play (Walters and Kitchell 2001). Higher production rates during recoveries are sometimes predicted by ecosystem models with size- and age-structure effects, but only when such models include ecosystem-scale changes in productivity (regime shifts), such that trophically driven increases in production can drive population increases. We recommend that simple plots of SP rate vs. stock size be provided with stock assessment and ecosystem model results in general.

%\citep{{winkler2019jabbasel} Age-structured production models (ASPMs) are often preferred over biomass aggregated surplus production models (SPMs) as the former can track the propagation of cohorts and explicitly account for the effects of selective fishing, even in the absence of reliable size- or age data. Here, we introduce ‘JABBA-Select’, an extension of the JABBA software (Just Another Bayesian Biomass Assessment; Winker et al., 2018), that is able to overcome some of the shortcomings of conventional SPMs and allows a direct comparison to ASPMs. JABBASelect incorporates life history parameters and fishing selectivity and distinguishes between exploitable biomass (used to fit indices given fishery selectivity) and spawning biomass (used to predict surplus production). Applying JABBA-Select involves using an age-structured equilibrium model to convert the input parameters into multivariate normal priors for surplus-production productivity parameters. We illustrate the main elements of JABBA-Select using the stock parameters of South African silver kob (Argyrosomus inodorus, Scienidae) as a case study. This species is exploited by multiple fisheries and was selected as an example of a data moderate fishery that features strong contrast in selectivity over time and across fleets. For proof-of-concept, we use an agestructured simulation framework to compare the performance of JABBA-Select to: 1) a conventional Bayesian state-space model using a Pella-Tomlinson (PT) production function, 2) an ASPM with deterministic recruitment; and 3) an ASPM with stochastic recruitment. The PT model produced biased estimates of relative and absolute spawning biomass trajectories and associated reference points, which by contrast could be fairly accurately estimated by JABBA-Select. JABBA-Select performed at least as well as the ASPMs in accuracy for most of the performance metrics and best characterized the stock status uncertainty. The results indicate that JABBA-Select is able to accurately account for moderate changes in selectivity and fleet dynamics over time and to provide a robust tool for data-moderate stock assessments.

\section{Acknowledgement}


\clearpage
%\bibliography{/home/laurence-kell/Desktop/refs.bib}
%\bibliographystyle{apalike}
\input{draft.bbl}

\clearpage
\section{Figures}


\newpage
\begin{figure}[h]
\centering
\includegraphics[width=\textwidth]{pe-tsmsy-1.png}
\caption{Time series of SSB, biomass and catch relative to $MSY$ based reference points.}
\label{fig:ts}
\end{figure}


\newpage
\begin{figure}[h]
\centering
\includegraphics[width=\textwidth]{pe-sp2-1.png}
\caption{Surplus production plotted against biomass, dark to light trajectory colours indicates early to late period; clockwise loops in surplus production indicate positive production anomalies.}
\label{fig:sp}
\end{figure}


\newpage
\begin{figure}[h]
\centering
\includegraphics[width=\textwidth]{pe-pf2-1.png}
\caption{Production functions, with trajectories of catch and biomass; coloured regions correspond to Kobe quadrants, i.e. red indicates overfishing and overfished.}
\label{fig:pf}
\end{figure}


\newpage
\begin{figure}[h]
\centering
\includegraphics[width=\textwidth]{pe-pe-1.png}
\caption{Time series of process error.}
\label{fig:pe}
\end{figure}


\newpage
\begin{figure}[h]
\centering
\includegraphics[width=\textwidth]{pe-pe2-1.png}
\caption{Process error, scaled relative to mean biomass, with regimes.}
\label{fig:pe2}
\end{figure}


\end{document}

Update B-ratio and/or F-ratio values from recent assessments and deal with F0.1 issue.

2. Retained Species: Assessed
2.1. Objective
Update B-ratio and/or F-ratio values from recent assessments and deal with F0.1 issue.


12. Environmental Pressure
12.1. Objective
Create an indicator based on impact of habit on fisheries



\clearpage
\section{Figures}


\newpage
\begin{figure}[h]
\centering
\includegraphics[width=\textwidth]{pe-tsmsy-1.png}
\caption{Time series of SSB, biomass and catch relative to $MSY$ based reference points.}
\label{fig:ts}
\end{figure}


\newpage
\begin{figure}[h]
\centering
\includegraphics[width=\textwidth]{pe-sp2-1.png}
\caption{Surplus production plotted against biomass, dark to light trajectory colours indicates early to late period; clockwise loops in surplus production indicate positive production anomalies.}
\label{fig:sp}
\end{figure}


\newpage
\begin{figure}[h]
\centering
\includegraphics[width=\textwidth]{pe-pf2-1.png}
\caption{Production functions, with trajectories of catch and biomass; coloured regions correspond to Kobe quadrants, i.e. red indicates overfishing and overfished.}
\label{fig:pf}
\end{figure}


\newpage
\begin{figure}[h]
\centering
\includegraphics[width=\textwidth]{pe-pe-1.png}
\caption{Time series of process error.}
\label{fig:pe}
\end{figure}


\newpage
\begin{figure}[h]
\centering
\includegraphics[width=\textwidth]{pe-pe2-1.png}
\caption{Process error, scaled relative to mean biomass, with regimes.}
\label{fig:pe2}
\end{figure}


\end{document}

Update B-ratio and/or F-ratio values from recent assessments and deal with F0.1 issue.

2. Retained Species: Assessed
2.1. Objective
Update B-ratio and/or F-ratio values from recent assessments and deal with F0.1 issue.


12. Environmental Pressure
12.1. Objective
Create an indicator based on impact of habit on fisheries



\clearpage
\section{Figures}


\newpage
\begin{figure}[h]
\centering
\includegraphics[width=\textwidth]{pe-tsmsy-1.png}
\caption{Time series of SSB, biomass and catch relative to $MSY$ based reference points.}
\label{fig:ts}
\end{figure}


\newpage
\begin{figure}[h]
\centering
\includegraphics[width=\textwidth]{pe-sp2-1.png}
\caption{Surplus production plotted against biomass, dark to light trajectory colours indicates early to late period; clockwise loops in surplus production indicate positive production anomalies.}
\label{fig:sp}
\end{figure}


\newpage
\begin{figure}[h]
\centering
\includegraphics[width=\textwidth]{pe-pf2-1.png}
\caption{Production functions, with trajectories of catch and biomass; coloured regions correspond to Kobe quadrants, i.e. red indicates overfishing and overfished.}
\label{fig:pf}
\end{figure}


\newpage
\begin{figure}[h]
\centering
\includegraphics[width=\textwidth]{pe-pe-1.png}
\caption{Time series of process error.}
\label{fig:pe}
\end{figure}


\newpage
\begin{figure}[h]
\centering
\includegraphics[width=\textwidth]{pe-pe2-1.png}
\caption{Process error, scaled relative to mean biomass, with regimes.}
\label{fig:pe2}
\end{figure}


\end{document}

Update B-ratio and/or F-ratio values from recent assessments and deal with F0.1 issue.

2. Retained Species: Assessed
2.1. Objective
Update B-ratio and/or F-ratio values from recent assessments and deal with F0.1 issue.


12. Environmental Pressure
12.1. Objective
Create an indicator based on impact of habit on fisheries



\clearpage
\section{Figures}


\newpage
\begin{figure}[h]
\centering
\includegraphics[width=\textwidth]{pe-tsmsy-1.png}
\caption{Time series of SSB, biomass and catch relative to $MSY$ based reference points.}
\label{fig:ts}
\end{figure}


\newpage
\begin{figure}[h]
\centering
\includegraphics[width=\textwidth]{pe-sp2-1.png}
\caption{Surplus production plotted against biomass, dark to light trajectory colours indicates early to late period; clockwise loops in surplus production indicate positive production anomalies.}
\label{fig:sp}
\end{figure}


\newpage
\begin{figure}[h]
\centering
\includegraphics[width=\textwidth]{pe-pf2-1.png}
\caption{Production functions, with trajectories of catch and biomass; coloured regions correspond to Kobe quadrants, i.e. red indicates overfishing and overfished.}
\label{fig:pf}
\end{figure}


\newpage
\begin{figure}[h]
\centering
\includegraphics[width=\textwidth]{pe-pe-1.png}
\caption{Time series of process error.}
\label{fig:pe}
\end{figure}


\newpage
\begin{figure}[h]
\centering
\includegraphics[width=\textwidth]{pe-pe2-1.png}
\caption{Process error, scaled relative to mean biomass, with regimes.}
\label{fig:pe2}
\end{figure}


\end{document}

Update B-ratio and/or F-ratio values from recent assessments and deal with F0.1 issue.

2. Retained Species: Assessed
2.1. Objective
Update B-ratio and/or F-ratio values from recent assessments and deal with F0.1 issue.


12. Environmental Pressure
12.1. Objective
Create an indicator based on impact of habit on fisheries

